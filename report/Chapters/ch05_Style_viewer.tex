% TBD by Wolfgang
\section{Style Viewers} 
\label{sec:viewers}
As we have shown earlier, the style views are the essence of the creative process so it's worth looking at this in its raw form. That's why we included some style viewers into the project to apply to this task. In this section we introduce them.

\subsection{Text Based viewer}
\label{sec:viewers.text}
As discussed earlier the styles are saved in a JSON file. Even though this format is made to be readable by humans we thought of a more pleasant way to view the style while developing. So we gave the style objects a print method. This method will print the style onto the current output of your python session, as can be seen in the figure below:
\begin{figure}[ht]
\centering
\includegraphics[scale=1]{Chapters/pic/text_print.png}
\caption{A sample chord printed with the print method}
\end{figure}

This figure shows the dummy style. The viewer orders all the style sequences according to their likelihood, starting with the most likely. It also translates our numeric representation into a tonal representation for a given key (default key is C major).

That way we can easily see how the style will look like and if after applying the creative module an odd thing happened to the style, that would be harder to spot after the actual harmonizer. 

\subsection{MusicXML Creator}
\label{sec:viewers.musicxml}
Also faster and more visual would be to translate the style into a MusicXML file and view it with your favorite MusicXML viewer. We used MuseScore for that purpose. 

\begin{figure}[ht]
\includegraphics[scale=.5]{Chapters/pic/xml_print.png}
\caption{MusicXML of a sample style viewed with the MuseScore program}
\end{figure}

As seen in the figure above we used the write MusicXML method of the visualizer. It does the same thing as the print method, that it orders the chord sequences starting with the most likely. But instead of printing it to the output it creates a MusicXML file. For convenience we defined a bunch of default values, namely the key is C major, the duration of a tone is quarter pitch, as well as we use a four quarters rhythm. After every sequence we inserted a quarter break to indicate the end of a harmonization sequence.

With this visualization you have the major advantage that now you have standardized data structure and you are able to process them further. But mainly you want to view it in a score notation or listen to your creation.

\subsection{Extempore}
\label{sec:viewers.extempore}

TODO: Christoph (see also Section \ref{sec:tools.extempore})