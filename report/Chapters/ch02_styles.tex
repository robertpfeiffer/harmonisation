
\subsection{Harmonic Style}
Harmonic styles in the scope of this project are defined in terms of a name, a description and a list of weighted chord progressions (see Chord Progressions below). This is a strong simplification of the usual notion of a musical style, but thereby reduces complexity and allows for formalization of an otherwise abstract concept.

Harmonic styles are implemented by the class \texttt{HarmonicStyle}, and specific manifestations are represented in terms of one JSON file (see JSON Format below) for each distinct style. With the help of static methods of the \texttt{HarmonicStyleIO} class, an instance of \texttt{HarmonicStyle} can be created from or serialized into a JSON file.

\subsubsection{Chord Progressions}
A chord progression is a list of two or more chords in fixed order. These chords are represented as a list of chromatic notes within one octave relative to an unknown reference note (e.g. the root of the melody's key). For example, the simple tonic could be expressed as (0,4,7). A perfect cadence could then be expressed as [(0,4,7), (2,7,11), (0,4,7)]. The advantage of this relative definition is that it is independent of interpretation or musical function of specific notes and degree of the scale.

Actual chords can be obtained from these descriptions by defining a reference note, e.g. for the reference note C the former chord (0,4,7) would be equivalent to a chord formed by the notes C, E, G. Note that the given numbers do not indicate the octave or order of the given note. While the chord (2,7,11) stands for D, G, and B (for reference note C), the note G could also be realized as the bass note. This is implemented as a normalization of the numbers into the range of (0,11), shifting all notes into the same octave. After normalization, the notes are stored as sets, i.e. each note can occur only once per chord.

Each chord progression in a harmonic style, characterized by its list of notes, additionally holds a weight. Its weight indicates how much it is preferred over other chord progressions of the harmonic style, or specify the relative frequency of this progression in harmonizations developed from this specific harmonic style. These weights can be any positive integer, but have to be consistent within the same harmonic style.

\subsubsection{JSON Format}
The JSON (short for: JavaScript Object Notation) format \cite{json} is a light-weight, language-independent notation for data exchange, often used in client-server-applications. Data in the JSON format consists of attribute-value pairs and can therefore easily be translated into associative arrays (e.g. dictionaries in Python). Utilization of this format has many advantages: It is human-readable and is used by many modern computer programs and web-applications. Therefore, by using the JSON format, we can make use of the libraries and interfaces programming languages and other software (e.g. Music21) provides. An example of a chord progression defined in JSON format can be seen in Fig. \ref{fig:jsonex}.

\begin{figure}
\centering
\begin{lstlisting}[frame=single]
{
	"description" : "T->T",
           "weight": 1,
           "chords": [
                [0,4,7],
                [0,4,7]
           ]
}
\end{lstlisting}
\caption{Example of a progression defined in JSON}
\label{fig:jsonex}
\end{figure}